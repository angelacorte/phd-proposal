\documentclass[12pt, a4paper]{article}
%\usepackage{natbib}
\usepackage{cite}
\usepackage{amsmath,amssymb,amsfonts}
\usepackage{algorithmic}
\usepackage{graphicx}
\usepackage{textcomp}
\usepackage{xcolor}
\usepackage[inline]{enumitem}
\usepackage[hidelinks]{hyperref}

\usepackage{myacronyms}
\usepackage{listings}
\usepackage{cleveref}

\begin{document}

\title{Titolo accattivante}
\author{Angela Cortecchia}
\date{\today}
\maketitle

\setlength{\parindent}{0em}
\setlength{\parskip}{1em}

% ----------------------------------------
\section{State of the art}\label{sec:state-of-the-art}

\paragraph{Collective Adaptive Systems}

\ac{cas} are systems composed of multiple autonomous entities, such as devices, sensors and actuators,
that interact to achieve a common goal~\cite{ferscha2015}.
%
These systems are known for their ability to adjust to changes in the environment,
systems requirements, or operational conditions.

\ac{cas} are frequently used in \ac{cps} applications,
where devices collaborate to achieve a common goal,
such as monitoring and controlling a physical environment or providing a service to users.
%
The coordination of devices in \ac{cas} can be challenging due to the heterogeneity of the devices involved,
resource constraints, and the need to adapt to dynamic changes in the environment.

Macroprogramming can be used to promote collective behaviors,
leveraging high-level abstractions and constructs to facilitate global coordination,
decentralized control, and adaptability in complex systems.


\paragraph{Macroprogramming}
The term \emph{macroprogramming}~\cite{casadei2023} refers to the concept of expressing the macroscopic behavior of a system
through a single program,
usually leveraging on macro-level abstractions.
%
This paradigm is driven by the need to capture \emph{system-level behavior} while abstracting the behavior and interaction
of the individual components involved.
%
Macroprogramming approaches have been suggested to simplify the development of systems involving many interconnected sensors,
actuators, and smart devices;
they can be applied in context like \ac{iot} and \ac{cps}.

Macroprogramming abstractions can promote collective behavior properties,
such as self-organizing or self-configuring in the context of \ac{cas}.
%
By declaring tasks within a specific spatio-temporal region,
systems can self-organize and effectively perform the task at hand,
allowing for dynamic adaptation to the current deployment and spatial position of the components involved.

\paragraph{Self-organization}
Coordination models are based on the notion that interaction among multiple independent and autonomous software systems
can be designed as a space orthogonal to pure computation.
%
This idea can be reified into a concept of shared data space, enabling so-called \emph{generative communication}.

\paragraph{Field Calculus}

\paragraph{Aggregate Programming}

\paragraph{Actual tools}

% coerenza con il percorso formativo

% ----------------------------------------
\section{Project description}\label{sec:project-description}


% ----------------------------------------
\section{Expected results}\label{sec:expected-results}


% ----------------------------------------
\section{Subdivision of the project and timeline}\label{sec:subdivision-of-the-project-and-timeline}


% ----------------------------------------
\section{Proposed evaluation criteria}\label{sec:proposed-evaluation-criteria}


% ----------------------------------------
\bibliographystyle{IEEEtran}
\bibliography{bibliography}

\end{document}
