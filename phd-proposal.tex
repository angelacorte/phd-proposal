\documentclass[12pt, a4paper]{article}
%\usepackage{natbib}
\usepackage{cite}
\usepackage{amsmath,amssymb,amsfonts}
\usepackage{algorithmic}
\usepackage{graphicx}
\usepackage{textcomp}
\usepackage{xcolor}
\usepackage[inline]{enumitem}
\usepackage[hidelinks]{hyperref}

\usepackage{myacronyms}
\usepackage{listings}

\newenvironment{inlinelist}{\begin{enumerate*}[label=\emph{(\roman*)}]}{\end{enumerate*}}

\usepackage{cleveref}

\begin{document}

\title{Titolo accattivante}
\author{Angela Cortecchia}
\date{\today}
\maketitle

\setlength{\parindent}{0em}
\setlength{\parskip}{1em}

% ----------------------------------------
\section{State of the art}\label{sec:state-of-the-art}
%
\paragraph{Collective Adaptive Systems}

\ac{cas} are systems composed of multiple autonomous entities, such as devices, sensors and actuators,
that interact to achieve a common goal~\cite{ferscha2015}.
%
These systems are known for their ability to adjust to changes in the environment,
systems requirements, or operational conditions.

\ac{cas} are frequently used in \ac{cps} applications,
where devices collaborate to achieve a common goal,
such as monitoring and controlling a physical environment or providing a service to users.
%
The coordination of devices in \ac{cas} can be challenging due to the heterogeneity of the devices involved,
resource constraints, and the need to adapt to dynamic changes in the environment.

Macroprogramming can be used to promote collective behaviors,
leveraging high-level abstractions and constructs to facilitate global coordination,
decentralized control, and adaptability in complex systems.


\paragraph{Macroprogramming}
The term \emph{macroprogramming}~\cite{casadei2023} refers to the concept of expressing the macroscopic behavior of a system
through a single program,
usually leveraging on macro-level abstractions.
%
This paradigm is driven by the need to capture \emph{system-level behavior} while abstracting the behavior and interaction
of the individual components involved.
%
Macroprogramming approaches have been suggested to simplify the development of systems involving many interconnected sensors,
actuators, and smart devices;
they can be applied in context like \ac{iot} and \ac{cps}.

Macroprogramming abstractions can promote collective behavior properties,
such as self-organizing or self-configuring in the context of \ac{cas}.
%
By declaring tasks within a specific spatio-temporal region,
systems can self-organize and effectively perform the task at hand,
allowing for dynamic adaptation to the current deployment and spatial position of the components involved.

\paragraph{Self-organization}
Coordination models are based on the notion that interaction among multiple independent and autonomous software systems
can be designed as a space orthogonal to pure computation.
%
This idea can be reified into a concept of shared data space, enabling so-called \emph{generative communication}.

Over the course of time, different approaches have been created,
such as ``Linda''~\cite{ViroliCoordination2012} and ``MARS''~\cite{mars},
suggesting innovative techniques for programming systems with devices of different nature,
focusing on the coordination of centralized local components,
but not on the distribution of the system.
%
The main problems that can be encountered in distributed systems are dealing with
\begin{inlinelist}
    \item openness, as unexpected environment changes,
    \item large scale of agents and coordination abstractions to be managed,
    \item intrinsic adaptiveness, such as the ability to intercept relevant events and react to them,
    guaranteeing the resilience of the system.
\end{inlinelist}

The challenges require a \emph{self-organizing and coordination} approach,
wherein coordination abstractions solely manage logical interactions.
%
This ensures the emergence of global and robust patterns of correct coordination behavior.

\paragraph{Field Calculus}
To facilitate self-organization patterns of agents in complex environments,
the concept of \emph{coordination field} has been introduced.
%
This abstraction serves as a navigational tool for agents over the actual environment.

In this context, the tuple-based middleware \textit{TOTA} (Tuples On The Air)~\cite{tota} has been suggested
to support field-based coordination for pervasive-computing applications.

Somewhat independently of the challenge of identifying appropriate coordination models for distributed and situated systems,
several studies have tackled analogous issues in the broader endeavor of constructing distributed intelligent systems.
%
This involves promoting higher abstractions of spatial collective adaptive systems.

Among the studies such as for managing space-time computing models for the manipulation of distributed data structures,
the notion of computational fields was proposed~\cite{JLAMP2019}.
%
Consequently, the \ac{fc} was proposed as a foundational model for the coordination of computational devices spread in physical environments,
also known as \ac{ac}.
%
\ac{fc} was introduced as a minimal core calculus with the aim of capturing the fundaments that make use of computational fields:
functions over and with fields, their evolution over time and the construction of field of values from neighbors.

The main concept of \ac{fc} is to specify the aggregate system behavior of a network of devices,
where devices that can directly communicate with each other are indicated through a dynamic network relation.
%
An example of its application is within a sensor network with a range of a broadcast communication.

%todo more on fc

\paragraph{Aggregate Programming}

\paragraph{Actual tools}

% coerenza con il percorso formativo

% ----------------------------------------
\section{Project description}\label{sec:project-description}


% ----------------------------------------
\section{Expected results}\label{sec:expected-results}


% ----------------------------------------
\section{Subdivision of the project and timeline}\label{sec:subdivision-of-the-project-and-timeline}


% ----------------------------------------
\section{Proposed evaluation criteria}\label{sec:proposed-evaluation-criteria}


% ----------------------------------------
\bibliographystyle{IEEEtran}
\bibliography{bibliography}

\end{document}
