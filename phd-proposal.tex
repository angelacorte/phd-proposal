\documentclass[12pt]{article}
\usepackage{mgates-letter}
\definecolor{dark_blue} {rgb}{0., 0., 0.65}

\usepackage{textcomp}
\usepackage{mathrsfs}  % mathscr font
\usepackage{boxedminipage}
\usepackage{rotating}
%\usepackage{natbib}
\usepackage[colorlinks, filecolor=dark_blue, urlcolor=dark_blue, linkcolor=black, citecolor=black]{hyperref}
\usepackage{cite}
\usepackage{amsmath,amssymb,amsfonts}
\usepackage{graphicx}
\usepackage[dvipsnames]{xcolor}
\def\BibTeX{{\rm B\kern-.05em{\sc i\kern-.025em b}\kern-.08em
T\kern-.1667em\lower.7ex\hbox{E}\kern-.125emX}}
\usepackage[inline]{enumitem}
\usepackage{myacronyms}
\usepackage{listings}
\usepackage{mathtools}
\usepackage{stmaryrd}
\usepackage{subcaption}
\usepackage{nicefrac}
\usepackage{float}
\usepackage{cleveref}

\begin{document}

\title{Titolo accattivante}
\author{Angela Cortecchia}
\date{\today}
\maketitle

\setlength{\parindent}{0em}
\setlength{\parskip}{1em}

% ----------------------------------------
\section{State of the art}\label{sec:state-of-the-art}

\paragraph{Collective Adaptive Systems}

\ac{cas} are systems composed of multiple autonomous entities, such as devices, sensors and actuators,
that interact to achieve a common goal~\cite{ferscha2015}.
%
These systems are known for their ability to adjust to changes in the environment,
systems requirements, or operational conditions.

\ac{cas} are frequently used in \ac{cps} applications,
where devices collaborate to achieve a common goal,
such as monitoring and controlling a physical environment or providing a service to users.
%
The coordination of devices in \ac{cas} can be challenging due to the heterogeneity of the devices involved,
resource constraints, and the need to adapt to dynamic changes in the environment.

Macroprogramming can be used to promote collective behaviors,
leveraging high level abstractions and constructs to facilitate global coordination,
decentralized control, and adaptability in complex systems.


\paragraph{Macroprogramming}

\paragraph{Field Calculus}

\paragraph{Aggregate Programming}

\paragraph{Actual tools}

% coerenza con il percorso formativo

% ----------------------------------------
\section{Project description}\label{sec:project-description}


% ----------------------------------------
\section{Expected results}\label{sec:expected-results}


% ----------------------------------------
\section{Subdivision of the project and timeline}\label{sec:subdivision-of-the-project-and-timeline}


% ----------------------------------------
\section{Proposed evaluation criteria}\label{sec:proposed-evaluation-criteria}


% ----------------------------------------
\bibliographystyle{IEEEtran}
\bibliography{bibliography}

\end{document}
